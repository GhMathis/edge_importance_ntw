% Options for packages loaded elsewhere
\PassOptionsToPackage{unicode}{hyperref}
\PassOptionsToPackage{hyphens}{url}
%
\documentclass[
]{article}
\usepackage{amsmath,amssymb}
\usepackage{lmodern}
\usepackage{iftex}
\ifPDFTeX
  \usepackage[T1]{fontenc}
  \usepackage[utf8]{inputenc}
  \usepackage{textcomp} % provide euro and other symbols
\else % if luatex or xetex
  \usepackage{unicode-math}
  \defaultfontfeatures{Scale=MatchLowercase}
  \defaultfontfeatures[\rmfamily]{Ligatures=TeX,Scale=1}
\fi
% Use upquote if available, for straight quotes in verbatim environments
\IfFileExists{upquote.sty}{\usepackage{upquote}}{}
\IfFileExists{microtype.sty}{% use microtype if available
  \usepackage[]{microtype}
  \UseMicrotypeSet[protrusion]{basicmath} % disable protrusion for tt fonts
}{}
\makeatletter
\@ifundefined{KOMAClassName}{% if non-KOMA class
  \IfFileExists{parskip.sty}{%
    \usepackage{parskip}
  }{% else
    \setlength{\parindent}{0pt}
    \setlength{\parskip}{6pt plus 2pt minus 1pt}}
}{% if KOMA class
  \KOMAoptions{parskip=half}}
\makeatother
\usepackage{xcolor}
\usepackage[margin=1in]{geometry}
\usepackage{graphicx}
\makeatletter
\def\maxwidth{\ifdim\Gin@nat@width>\linewidth\linewidth\else\Gin@nat@width\fi}
\def\maxheight{\ifdim\Gin@nat@height>\textheight\textheight\else\Gin@nat@height\fi}
\makeatother
% Scale images if necessary, so that they will not overflow the page
% margins by default, and it is still possible to overwrite the defaults
% using explicit options in \includegraphics[width, height, ...]{}
\setkeys{Gin}{width=\maxwidth,height=\maxheight,keepaspectratio}
% Set default figure placement to htbp
\makeatletter
\def\fps@figure{htbp}
\makeatother
\setlength{\emergencystretch}{3em} % prevent overfull lines
\providecommand{\tightlist}{%
  \setlength{\itemsep}{0pt}\setlength{\parskip}{0pt}}
\setcounter{secnumdepth}{-\maxdimen} % remove section numbering
\ifLuaTeX
  \usepackage{selnolig}  % disable illegal ligatures
\fi
\IfFileExists{bookmark.sty}{\usepackage{bookmark}}{\usepackage{hyperref}}
\IfFileExists{xurl.sty}{\usepackage{xurl}}{} % add URL line breaks if available
\urlstyle{same} % disable monospaced font for URLs
\hypersetup{
  hidelinks,
  pdfcreator={LaTeX via pandoc}}

\author{}
\date{\vspace{-2.5em}}

\begin{document}

\hypertarget{resaux-dinteractions-huxf4tes-virus-quantification-de-limportance-des-association-et-identification-des-interactions-structurantes.}{%
\section{Resaux d'interactions Hôtes-Virus : Quantification de
l'importance des association, et identification des interactions
structurantes.}\label{resaux-dinteractions-huxf4tes-virus-quantification-de-limportance-des-association-et-identification-des-interactions-structurantes.}}

\hypertarget{introduction}{%
\subsection{Introduction}\label{introduction}}

La prédiction d'interaction potentiel entre hôte virus \(\rightarrow\)
challenge important

Prédire des interaction entre espèces = difficile

Grosse quantité de données mais pas standardisée + beaucoup de biais

Besoin de méthodes qui prennent en compte ce contrainte

Différentes approches : - a partir des traits (plus facile pour
généraliser) - a partir des réseaux interactions

Des méthodes pour prédire les associations entre espèce commence a faire
apparition Besoin de s'assurer que ces méthodes ne prédisent pas
seulement les interactions fortement structurante du réseaux d'une part
Besoin de connaitre quelle interaction sont structurantes et lesquelles
le sont moins afin d'orienter les études Peut potentiellement amélioré
la prédiction d'interaction graçe a une meilleur connaissance des résaux

\hypertarget{mat-et-meth}{%
\subsection{Mat et meth}\label{mat-et-meth}}

Plusieur approches : - Perturbations du résaux puis comparaison des
Standard Value Décomposition + Random Graph Dot Product

\begin{itemize}
\tightlist
\item
  Perturbations puis annalyse spectral et mesure communicabilité
\end{itemize}

\[G_{pq}=\sum^n_{j=1}\varphi_j(p)\varphi_j(q)e^{\lambda_j}\]

\begin{itemize}
\tightlist
\item
  Stockastique graphs models - Stockastick block models - Latent block
  models
\item
  Approche exploratoire de backpropagation ?
\end{itemize}

\hypertarget{ruxe9sultats}{%
\subsection{Résultats}\label{ruxe9sultats}}

\hypertarget{disscution}{%
\subsection{Disscution}\label{disscution}}

\hypertarget{conclusion}{%
\subsection{Conclusion}\label{conclusion}}

\end{document}
